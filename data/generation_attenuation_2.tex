\begin{tabular}{ll}
 & References \\
Component en &  \\
1,1,1-Trichloroethane & \cite{Ormad2008-bd},\cite{Bhattacharya_1996},\cite{Wu_2006},\cite{Khan_2019} \\
\cline{1-2}
1,2-Dichloroethane & nan \\
\cline{1-2}
DEHP & \cite{Kim2009-ik} \\
\cline{1-2}
Alachlor & \cite{Ormad_2008},\cite{RAHMANI201882},\cite{Gorito_2021},\cite{Stamatis_2010} \\
\cline{1-2}
Aldrin & \cite{Ormad_2008},\cite{RAHMANI201882},\cite{Katsoyiannis_2004} \\
\cline{1-2}
Anthracene & \cite{Teijon_2010} \\
\cline{1-2}
Arsenic & \cite{Feng_2018},\cite{Harper_1992},\cite{Drozdova_2018},\cite{Wu_2022},\cite{Balasubramanian_2001},\cite{Salehi_2020},\cite{Zhou_2018},\cite{Busetti_2005},\cite{Choubert_2011},\cite{Mass__1995} \\
\cline{1-2}
Atrazine & \cite{Teijon_2010},\cite{Mailler_2013},\cite{Stamatis_2010},\cite{RAHMANI201882},\cite{Hai_2020},\cite{Shahmahdi2020-ok},\cite{Gorito_2021},\cite{Ormad_2008},\cite{Scheideler_2011},\cite{Zhang_2021},\cite{Hijosa_Valsero_2013},\cite{Tkachenko2020-sk},\cite{Kruithof_2007} \\
\cline{1-2}
Benzo(b) fluoride & \cite{Teijon_2010} \\
\cline{1-2}
Cadmium & \cite{Lester_1979},\cite{Li_2010},\cite{da_Silva_Oliveira_2007},\cite{SORME_2002},\cite{Yoshida_2015},\cite{Mass__1995},\cite{Buzier_2006},\cite{Poon_1986},\cite{Busetti_2005},\cite{Goldstone_1990},\cite{Choubert_2011},\cite{Moriyama_1988},\cite{Drozdova_2018},\cite{Honarmandrad_2020},\cite{Sterritt_1981},\cite{Brown_1973},\cite{Samarghandi_2007},\cite{Hargreaves_2016},\cite{Nieminski_1995},\cite{Karvelas_2003},\cite{Mahmoudkhani_2014},\cite{Lester_1983},\cite{Zhou_2018},\cite{OLIVER_1974},\cite{Sudo_1973},\cite{Gardner_2013} \\
\cline{1-2}
Carbon tetrachloride & \cite{Bhattacharya_1996} \\
\cline{1-2}
Chlorobenzene & \cite{Bhattacharya_1996} \\
\cline{1-2}
Chlorpyrifos & \cite{Ormad_2008},\cite{RAHMANI201882},\cite{Stamatis_2010} \\
\cline{1-2}
Chrome & \cite{Lester_1979},\cite{da_Silva_Oliveira_2007},\cite{Yoshida_2015},\cite{Feng_2018},\cite{Marzougui_2018},\cite{Buzier_2006},\cite{Orescanin_2013},\cite{Busetti_2005},\cite{Goldstone_1990},\cite{Choubert_2011},\cite{Stoveland_1979},\cite{Moriyama_1988},\cite{Drozdova_2018},\cite{Sterritt_1981},\cite{Brown_1973},\cite{Hargreaves_2016},\cite{Karvelas_2003},\cite{Lester_1983},\cite{Mansourri_2016},\cite{Zhou_2018},\cite{OLIVER_1974},\cite{Chen_2005},\cite{Edokpayi_2015},\cite{Sudo_1973},\cite{Gardner_2013} \\
\cline{1-2}
Ciprofloxacin & \cite{Kim_2009},\cite{Rahmani_2018},\cite{Feder_2015},\cite{Javid_2020},\cite{Coutu_2013},\cite{Brtnick__2019},\cite{Ortiz_de_Garc_a_2013},\cite{Castrignan__2020},\cite{Heaney_2015},\cite{Al_Maadheed_2019},\cite{K_mmerer_2000},\cite{Rizzo_2019},\cite{Tewari_2013},\cite{Arany_2014},\cite{Meng_2015},\cite{Alder_2004},\cite{Casta_o_Trias_2020},\cite{Garoma_2010},\cite{Miyazono_2015},\cite{Verlicchi_2014},\cite{Ros_n_2015},\cite{Garc_a_Rey_2006},\cite{Altenburger_2015},\cite{Uski_2015},\cite{Zhao_2015},\cite{K_mmerer_2003} \\
\cline{1-2}
Chloroalkanes & \cite{Mailler_2013} \\
\cline{1-2}
Clotrimazole & \cite{Liu_2017},\cite{Juksu_2019} \\
\cline{1-2}
Copper & \cite{Lester_1979},\cite{da_Silva_Oliveira_2007},\cite{SORME_2002},\cite{Yoshida_2015},\cite{Mass__1995},\cite{Feng_2018},\cite{Marzougui_2018},\cite{Johnson_2008},\cite{Buzier_2006},\cite{Gupta_1998},\cite{Orescanin_2013},\cite{Busetti_2005},\cite{Goldstone_1990},\cite{Choubert_2011},\cite{Pagliaccia_2022},\cite{Stoveland_1979},\cite{Moriyama_1988},\cite{Drozdova_2018},\cite{Honarmandrad_2020},\cite{Brown_1973},\cite{Sterritt_1981},\cite{Hargreaves_2016},\cite{Karvelas_2003},\cite{Mahmoudkhani_2014},\cite{Lester_1983},\cite{Mansourri_2016},\cite{Mailler_2013},\cite{Zhou_2018},\cite{Yuan_2020},\cite{OLIVER_1974},\cite{Edokpayi_2015},\cite{Sudo_1973},\cite{Gardner_2013} \\
\cline{1-2}
Dichlorobenzene & \cite{Mailler_2013} \\
\cline{1-2}
Dichlorvós & \cite{Stamatis_2010} \\
\cline{1-2}
Dicofol & \cite{Ormad_2008},\cite{RAHMANI201882} \\
\cline{1-2}
Dieldrin & \cite{Ormad_2008},\cite{RAHMANI201882},\cite{Mailler_2013},\cite{Katsoyiannis_2004} \\
\cline{1-2}
Diuron & \cite{Teijon_2010},\cite{Mailler_2013},\cite{RAHMANI201882},\cite{Ormad_2008},\cite{Kruithof2007-os},\cite{Kruithof_2007},\cite{Echevarr_a_2019} \\
\cline{1-2}
Endosulfan & \cite{Ormad_2008},\cite{RAHMANI201882},\cite{Katsoyiannis_2004} \\
\cline{1-2}
Endrin & \cite{Ormad_2008},\cite{RAHMANI201882},\cite{Katsoyiannis_2004} \\
\cline{1-2}
Ethylbenzene & \cite{Bhattacharya_1996},\cite{Shammay_2018} \\
\cline{1-2}
Fluconazole & \cite{Liu_2017},\cite{Juksu_2019} \\
\cline{1-2}
Fluoranthene & \cite{Teijon_2010} \\
\cline{1-2}
Fluorides & \cite{Onyango_2006} \\
\cline{1-2}
Heptachlor & \cite{Ormad_2008},\cite{RAHMANI201882},\cite{Katsoyiannis_2004} \\
\cline{1-2}
Hexabromocyclodecane & \cite{Ruan_2019},\cite{Ichihara_2014},\cite{van_Leeuwen_2008},\cite{Schecter_2010},\cite{Gao_2019},\cite{Kim_2016},\cite{Kim_2018},\cite{Eljarrat_2014},\cite{Roosens_2009} \\
\cline{1-2}
Hexachlorobutadiene & \cite{Teijon_2010},\cite{Oonnittan_2009},\cite{Mailler_2013},\cite{Katsoyiannis_2004},\cite{Ormad_2008} \\
\cline{1-2}
Imazalil & \cite{Arias_2014},\cite{Jim_nez_2014},\cite{ISHIWATA_2002},\cite{Campo_2013},\cite{Blasco_2006},\cite{Matsumoto_1994},\cite{Kne_evi__2012},\cite{Kahle_2008} \\
\cline{1-2}
Isodrin & \cite{Ormad_2008},\cite{RAHMANI201882} \\
\cline{1-2}
Isoproturon & \cite{Teijon_2010},\cite{Mailler_2013},\cite{Stamatis_2010},\cite{RAHMANI201882},\cite{Gorito_2021},\cite{Ormad_2008} \\
\cline{1-2}
Lead & \cite{Lester_1979},\cite{da_Silva_Oliveira_2007},\cite{Dimitrova_1998},\cite{Yoshida_2015},\cite{Feng_2018},\cite{Teijon_2010},\cite{Marzougui_2018},\cite{Johnson_2008},\cite{Buzier_2006},\cite{Gupta_1998},\cite{Orescanin_2013},\cite{Busetti_2005},\cite{Goldstone_1990},\cite{Choubert_2011},\cite{Moriyama_1988},\cite{Drozdova_2018},\cite{Honarmandrad_2020},\cite{Brown_1973},\cite{Samarghandi_2007},\cite{Sterritt_1981},\cite{Hargreaves_2016},\cite{Karvelas_2003},\cite{Lester_1983},\cite{Chiu_2016},\cite{Mansourri_2016},\cite{Zhou_2018},\cite{Mal_2021},\cite{OLIVER_1974},\cite{Edokpayi_2015},\cite{Sudo_1973},\cite{Gardner_2013} \\
\cline{1-2}
Mercury & \cite{Lester_1979},\cite{Bodaly_1998},\cite{da_Silva_Oliveira_2007},\cite{Yoshida_2015},\cite{Balogh_1995},\cite{Feng_2018},\cite{Teijon_2010},\cite{Busetti_2005},\cite{Goldstone_1990},\cite{Moriyama_1988},\cite{Drozdova_2018},\cite{Sterritt_1981},\cite{Brown_1973},\cite{Hargreaves_2016},\cite{Balogh_2008},\cite{Chiu_2016},\cite{Zhou_2018},\cite{Giraldo_2020},\cite{OLIVER_1974} \\
\cline{1-2}
Metolachlor & \cite{Ormad_2008},\cite{RAHMANI201882},\cite{Collivignarelli_2004} \\
\cline{1-2}
Miconazole & \cite{Liu_2017},\cite{Juksu_2019} \\
\cline{1-2}
Naphtalene & \cite{Teijon_2010} \\
\cline{1-2}
Nickel & \cite{Lester_1979},\cite{Yoshida_2015},\cite{Feng_2018},\cite{Teijon_2010},\cite{Marzougui_2018},\cite{Buzier_2006},\cite{Orescanin_2013},\cite{Busetti_2005},\cite{Goldstone_1990},\cite{Choubert_2011},\cite{Moriyama_1988},\cite{Drozdova_2018},\cite{Honarmandrad_2020},\cite{Hargreaves_2016},\cite{KATSOYIANNIS20042685},\cite{Karvelas_2003},\cite{Chen_2021},\cite{Lester_1983},\cite{Mansourri_2016},\cite{Mailler_2013},\cite{Zhou_2018},\cite{OLIVER_1974} \\
\cline{1-2}
Nonylphenols (4-) & \cite{Lema_2017},\cite{Mailler_2013},\cite{Echevarr_a_2019} \\
\cline{1-2}
Octylphenols & \cite{Mailler_2013} \\
\cline{1-2}
Pentachlorobenzene & \cite{Teijon_2010} \\
\cline{1-2}
p.p'-DTT & \cite{Ormad_2008},\cite{RAHMANI201882},\cite{Katsoyiannis_2004} \\
\cline{1-2}
Selenium & \cite{Okonji_2021},\cite{Ali_2021},\cite{Zhou_2018} \\
\cline{1-2}
Simazine & \cite{Teijon_2010},\cite{Mailler_2013},\cite{RAHMANI201882},\cite{Ormad_2008},\cite{Gorito_2021} \\
\cline{1-2}
Sulfamethoxazole & \cite{Phillips_2015},\cite{Bradshaw_2015},\cite{Hai_2011},\cite{Liu_2017},\cite{Kim_2009},\cite{Zhang_2021},\cite{Kruithof2007-os},\cite{Dan_A_2013},\cite{Coutu_2013},\cite{Chandurvelan_2015},\cite{Echevarr_a_2019},\cite{Ortiz_de_Garc_a_2013},\cite{Teijon_2010},\cite{G_mez_Ramos_2011},\cite{Zhang_2017},\cite{Cai_2018},\cite{Beltr_n_2008},\cite{ter_Laak_2014},\cite{Hai_2020},\cite{Dantas_2008},\cite{Larcher_2012},\cite{Rizzo_2019},\cite{Tewari_2013},\cite{Nasuhoglu_2011},\cite{Ramos_2015},\cite{Br_ckner_2020},\cite{Al_Aukidy_2012},\cite{Subedi_2017},\cite{Kim_2015},\cite{Gracia_Lor_2012},\cite{Hu_2015},\cite{Johnson_2017},\cite{Radjenovic2009-ud},\cite{Lan_Chun_2015},\cite{Escol_Casas_2015},\cite{Wei_2019},\cite{Javid2020-ch},\cite{Casta_o_Trias_2020},\cite{Kang_2018},\cite{Straub_2015},\cite{Mart_nez_Costa_2018},\cite{Yan_2014},\cite{Zhang_2016},\cite{Woznicki_2015},\cite{Garoma_2010},\cite{Marug_n_2020},\cite{Shahmahdi_2020},\cite{Yuan_2015},\cite{Nichols_2015},\cite{Staley_2015},\cite{Batt_2007},\cite{Navedo_2015},\cite{Song_2021},\cite{Carballa_2008},\cite{K_mmerer_2003} \\
\cline{1-2}
Terbutylazine & \cite{Ormad_2008},\cite{RAHMANI201882} \\
\cline{1-2}
Terbutryn & \cite{Ormad_2008},\cite{RAHMANI201882} \\
\cline{1-2}
Tetrachloroethylene & \cite{Mitra_2013},\cite{Tkachenko_2020},\cite{Mailler_2013},\cite{Sponza_2001},\cite{Bhattacharya_1996},\cite{Glaze_1987},\cite{ARZATE2020110265},\cite{Salehi2020-aa} \\
\cline{1-2}
Toluene & \cite{Shammay_2018},\cite{Mailler_2013} \\
\cline{1-2}
Trichlorethylene & \cite{ARZATE2020110265},\cite{Bhattacharya_1996},\cite{Tkachenko_2020},\cite{Mailler_2013} \\
\cline{1-2}
Trichloromethane & \cite{Samarghandi2007-uz},\cite{Shammay_2018},\cite{Mailler_2013} \\
\cline{1-2}
Trifluralin & \cite{Ormad_2008},\cite{RAHMANI201882},\cite{Teijon_2010} \\
\cline{1-2}
Trimethoprim & \cite{Liu_2015},\cite{Liu_2017},\cite{Ballesteros_C_novas_2015},\cite{Guillossou_2019},\cite{Kim_2009},\cite{Coutu_2013},\cite{Ortiz_de_Garc_a_2013},\cite{Kuang_2013},\cite{Cai_2018},\cite{Radjenovi__2009},\cite{GOBEL_2007},\cite{ter_Laak_2014},\cite{Hai_2020},\cite{Lebrun_2015},\cite{Tewari_2013},\cite{Omanovi__2015},\cite{Wu_2016},\cite{Krol_2015},\cite{Kuang2013-mo},\cite{Gracia_Lor_2012},\cite{Johnson_2017},\cite{Arzate_2020},\cite{Kaegi_2015},\cite{Escol_Casas_2015},\cite{Javid2020-ch},\cite{Plumlee_2012},\cite{Yan_2014},\cite{Novovic_2015},\cite{Mart_nez_Costa_2018},\cite{Zhang_2016},\cite{Yuan_2015},\cite{Marug_n_2020},\cite{Verlicchi_2014},\cite{Batt_2007},\cite{Carballa_2008},\cite{Meulepas_2015},\cite{K_mmerer_2003} \\
\cline{1-2}
Venlafaxine & \cite{Choubert2011-ic},\cite{Skees_2018},\cite{Lambropoulou_2017},\cite{R_a_G_mez_2011},\cite{Vazquez_Roig_2014},\cite{Giannakis_2017},\cite{GAROMA2010814},\cite{Teijon_2010},\cite{Lin_2021},\cite{ter_Laak_2014},\cite{Gorito_2021},\cite{Metcalfe_2009},\cite{Subedi_2017},\cite{Wu2022-sp},\cite{Gracia_Lor_2012},\cite{Arzate_2020},\cite{Subedi_2013},\cite{Subedi_2019},\cite{Escol_Casas_2015},\cite{Javid2020-ch},\cite{Casta_o_Trias_2020},\cite{Hollman_2018},\cite{Garc_a_Gal_n_2016},\cite{Schl_sener_2015},\cite{R_a_G_mez_2012} \\
\cline{1-2}
Xyloene & \cite{Shammay_2018},\cite{Ruan2019-vy},\cite{Mailler_2013} \\
\cline{1-2}
Zinc & \cite{Lester_1979},\cite{Wang_2021},\cite{da_Silva_Oliveira_2007},\cite{SORME_2002},\cite{Yoshida_2015},\cite{Feng_2018},\cite{Marzougui_2018},\cite{Orescanin_2013},\cite{Gupta_1998},\cite{Busetti_2005},\cite{Goldstone_1990},\cite{Choubert_2011},\cite{Stoveland_1979},\cite{Drozdova_2018},\cite{Brown_1973},\cite{Sterritt_1981},\cite{Hargreaves_2016},\cite{Nieminski_1995},\cite{Karvelas_2003},\cite{Lester_1983},\cite{Mansourri_2016},\cite{Mailler_2013},\cite{Zhou_2018},\cite{OLIVER_1974},\cite{Edokpayi_2015},\cite{Sudo_1973},\cite{Gardner_2013} \\
\cline{1-2}
Chlorfenvinphos & \cite{Stamatis_2010},\cite{RAHMANI201882},\cite{Ormad_2008},\cite{Hijosa_Valsero_2013},\cite{Tkachenko2020-sk} \\
\cline{1-2}
\end{tabular}
